% ============================================================================
% Резюме
% ============================================================================

\section*{РЕЗЮМЕ}
\addcontentsline{toc}{section}{РЕЗЮМЕ}

Настоящата курсова работа разглежда проектирането и реализацията на система за разпределено управление на конфигурации и политики с поддръжка на версиониране. Съвременните софтуерни системи, изградени по микросървисна архитектура и разположени в облачни среди, изискват способност за централизирано управление на конфигурациите с възможност за проследяване на промените, връщане към предишни версии и поддържка на множество среди.

\textbf{Основната цел} на проекта е създаването на производствена платформа, която да осигурява неизменяемо версиониране на JSON конфигурации с пълна поддръжка на branching, rollback и diff операции. Системата трябва да осигурява мулти-тенант изолация на данните и пълен одитен запис на всички операции.

\textbf{Теоретичната обосновка} на избора на технологии се базира на анализ на специфичните изисквания към системите за управление на конфигурации. Документно-ориентираните бази данни се оказват архитектурно подходящи за този домейн поради естественото съвпадение между JSON конфигурациите и документния модел. MongoDB е избрана като конкретна реализация, предоставяща необходимата производителност, гъвкавост на схемата и възможности за хоризонтално мащабиране.

\textbf{Практическата реализация} включва Node.js бекенд с Express.js уеб фреймуърк и MongoDB база данни. Системата реализира модел на неизменяеми версии, при който всяка версия на конфигурация се съхранява като отделен документ с SHA256 контролна сума. Поддържат се операции за създаване на версии, връщане към предишни състояния, разклоняване на конфигурации и сравнение между версии.

\textbf{Ключови функционалности} на системата включват управление на конфигурации с операции за създаване, четене, актуализиране и изтриване; версиониране с пълна история на промените и възможност за възстановяване на произволна версия; разклоняване за паралелна работа върху различни варианти на конфигурации; разлики между версии с детайлно сравнение на структурните промени; мулти-тенант архитектура с пълна изолация на данните между клиенти; одитен запис с регистриране на всички операции за съответствие.

\textbf{Резултатите} от разработката демонстрират, че използването на документно-ориентирана база данни за версиониране на конфигурации предоставя значителни предимства спрямо релационния подход, включително по-проста архитектура, по-добра производителност при четене и естествена поддръжка на йерархични структури.

\textbf{Ключови думи:} NoSQL, MongoDB, версиониране, конфигурации, мулти-тенант, JSON, Node.js, документно-ориентирани бази данни, разпределени системи.
