% ============================================================================
% Глава 1: Въведение
% ============================================================================

\section{ВЪВЕДЕНИЕ}

\subsection{Актуалност на проблема}

В съвременните разпределени софтуерни системи управлението на конфигурации представлява критично предизвикателство \cite{mongodb_definitive,newman2015microservices}. Микросървисната архитектура, облачните инфраструктури и динамичните runtime политики изискват способност за бързо променяне на конфигурацията без прекъсване на услугите. Традиционните подходи за съхранение на конфигурации чрез файлови системи, релационни бази данни или key-value хранилища страдат от фундаментални ограничения при опит за реализиране на пълноценно версиониране.

Конкретният бизнес проблем, който настоящият проект адресира, произтича от няколко често срещани сценария в съвременната софтуерна разработка, включително управление на feature flags \cite{feature_flags}. \textbf{Feature Flags} представляват механизъм, при който разработчици трябва да активират или деактивират функционалности в production среда без необходимост от redeploy на приложението. Това изисква възможност за моментално прилагане на конфигурационни промени с пълен контрол върху състоянието. \textbf{A/B Testing} налага поддръжка на различни конфигурации за различни потребителски сегменти, което изисква способност за управление на множество версии на една и съща конфигурация, действащи паралелно. \textbf{Infrastructure as Code} подходът изисква версиониране на инфраструктурни дефиниции с възможност за връщане към стабилни състояния при проблеми. \textbf{Policy Engines} изискват динамични правила за достъп и сигурност, които могат да се променят в реално време с възможност за одит на промените. \textbf{Multi-tenant SaaS} архитектурата изисква изолирани конфигурации за всеки клиент с пълна логическа и физическа сепарация на данните.

При всички тези сценарии се налагат общи изисквания към системата за управление на конфигурации: възможност за моментален rollback при откриване на проблем, поддръжка на пълна история на промените с информация за автор и времева марка, възможност за сравнение между различни версии с детайлно представяне на разликите, както и възможност за експериментиране чрез разклоняване на конфигурациите в отделни branch-ове.

\subsection{Цели и задачи на проекта}

Основната цел на разработената платформа е създаването на производствена система за управление на конфигурации, която осигурява неизменяемо версиониране на JSON конфигурации с пълна поддръжка на разклоняване, връщане към предишни състояния и операции за сравнение между версии.

Платформата трябва да осигурява \textbf{неизменяемо версиониране}, при което всяка версия на конфигурация се съхранява като неизменяем документ с криптографска контролна сума, гарантираща целостта на данните. Системата трябва да поддържа \textbf{разклоняване и връщане към предишни състояния}, позволявайки създаването на експериментални клонове за тестване на нови функционалности и моментално връщане към стабилни версии при необходимост. Необходима е \textbf{пълна логическа изолация на данни} между различни организации чрез мулти-тенант архитектура, осигуряваща сигурност и конфиденциалност. Системата трябва да поддържа \textbf{одитен запис} на всички операции с конфигурации, регистрирайки автор, времева марка и тип на операцията за целите на съответствие с регулаторни изисквания. Необходимо е \textbf{автоматично откриване на разлики} между версии с детайлно сравнение на структурните промени в JSON документите. Платформата трябва да осигурява \textbf{поддръжка на множество среди}, позволявайки едновременно управление на development, staging и production конфигурации.

\subsection{Обосновка на избора на нерелационна база данни}

Изборът на документно-ориентирана NoSQL база данни, конкретно MongoDB, се базира на детайлен анализ на специфичните изисквания към системите за управление на конфигурации \cite{mongodb_docs,sadalage2012nosql}. Традиционните релационни бази данни се оказват неподходящи за този домейн поради фундаментални архитектурни различия в моделите на данни.

\textbf{Полуструктурираният характер на данните} представлява основно предизвикателство за релационния модел. Конфигурациите са естествено JSON документи с произволна дълбочина и йерархия, като различни конфигурации могат да имат коренно различна структура. Релационният модел изисква или използване на EAV анти-патерн, или чести schema миграции при промяна на структурата, които водят до технически дълг и сложна поддръжка.

\textbf{Гъвкавостта на схемата} при документно-ориентираните бази позволява съхраняване на различни конфигурации с различна структура без нужда от промяна на database schema. Това е особено важно за системи с динамични конфигурации, където структурата може да се променя често без предизвестие.

\textbf{Версионирането чрез документи} се реализира естествено при документния модел, където всяка версия е представена като отделен документ. Този подход елиминира необходимостта от сложни join операции и осигурява атомарен достъп до цялата конфигурация като единица, което е критично за производителността при четене.

\textbf{Хоризонталното мащабиране} е заложено в архитектурата на MongoDB чрез нативна поддръжка на sharding. Това позволява разпределение на данни по tenantId, което е критично за мулти-тенант системи с голям брой клиенти и високи изисквания към производителността.

\textbf{Query операциите върху вложени структури} се реализират ефективно чрез MongoDB агрегационния pipeline, който позволява сложни аналитични заявки върху вложени JSON полета без нужда от денормализация на данните. Това е съществено предимство при работа със сложни конфигурационни документи.

\textbf{CAP теоремата} определя компромисите между консистентност, наличност и толерантност към разделяния при разпределени системи. За система за конфигурации, където четенията са много по-чести от писанията, MongoDB предлага отличен баланс с възможност за настройка на consistency level според конкретните изисквания.

Проектът демонстрира, че за специфичния домейн на версиониране на конфигурации документно-ориентираните NoSQL бази данни не представляват компромис, а са архитектурно оптималното решение \cite{kleppmann2017designing}.
