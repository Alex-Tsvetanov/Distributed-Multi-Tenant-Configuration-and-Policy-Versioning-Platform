% ============================================================================
% Глава 2: Теоретична и технологична обосновка
% ============================================================================

\section{ТЕОРЕТИЧНА И ТЕХНОЛОГИЧНА ОБОСНОВКА}

\subsection{Нерелационни бази данни}

\subsubsection{Основни характеристики на NoSQL}

Нерелационните бази данни, известни още като NoSQL, се отличават от релационния модел по няколко ключови характеристики, които ги правят подходящи за специфични сценарии на употреба \cite{sadalage2012nosql}.

\textbf{Липсата на фиксирана схема} представлява фундаментална разлика спрямо SQL базите. Докато в релационния модел схемата трябва да бъде дефинирана предварително и всички записи трябва да следват тази структура, NoSQL базите позволяват динамична структура на записите. Тази характеристика е особено ценна за системи с полуструктурирани данни, където различни записи могат да имат различни полета според специфичните нужди.

\textbf{Разпределеното съхранение} е заложено в архитектурата на NoSQL базите от самото начало. Вместо вертикално мащабиране чрез добавяне на ресурси към един сървър, тези бази са проектирани за хоризонтално мащабиране чрез разпределяне на данни между множество сървъри. Механизмът sharding позволява автоматично разпределение на данните според дефиниран ключ, осигурявайки висока наличност и отказоустойчивост \cite{sharding_pattern}.

\textbf{BASE моделът} заменя традиционния ACID модел при много NoSQL системи \cite{cap_theorem}. Това е съзнателен компромис, който жертва незабавната консистентност в полза на по-висока наличност при разпределени среди. BASE означава Basically Available, Soft state, Eventually consistent и е особено подходящ за приложения, където краткотрайната несъвместимост между реплики е приемлива в името на по-добрата производителност и наличност.

\subsubsection{Типове нерелационни бази данни}

Съществуват четири основни категории нерелационни бази данни, всяка със специфични силни страни и подходящи сценарии на употреба. Таблица 2.1 представя сравнителен преглед на категориите.

\begin{table}[H]
\centering
\caption{Категории NoSQL бази данни}
\begin{tabular}{|p{4.5cm}|p{3.5cm}|p{5cm}|}
\hline
\textbf{Тип} & \textbf{Представители} & \textbf{Подходящи сценарии} \\
\hline
Документно-ориентирани & MongoDB, CouchDB, RavenDB & JSON документи, йерархични данни \\
\hline
Key-Value & Redis, DynamoDB, Riak & Кеширане, сесии, бърз достъп \\
\hline
Колонно-ориентирани & Cassandra, HBase, Bigtable & Time-series, аналитика \\
\hline
Графови & Neo4j, ArangoDB, OrientDB & Релации, препоръчителни системи \\
\hline
\end{tabular}
\end{table}

Документно-ориентираните бази съхраняват данни като JSON или BSON документи с вложена структура, което ги прави идеални за полуструктурирани данни. Key-Value хранилищата предоставят най-простия модел за бърз достъп по ключ, често използван за кеширане. Колонно-ориентираните бази оптимизират записа и четенето на големи обеми данни чрез съхраняване по колони вместо по редове. Графовите бази са специализирани за данни със сложни връзки между обекти.

\subsection{Избор на MongoDB}

За настоящия проект е избрана MongoDB като конкретна реализация на документно-ориентирана база данни. MongoDB съхранява данни в BSON формат, който представлява бинарна сериализация на JSON с допълнителна поддръжка за типове данни като Date и Binary \cite{mongodb_definitive,mongodb_docs}.

\subsubsection{Защо документно-ориентиран модел}

Конфигурационните данни са естествено документи с характеристики, които съвпадат идеално с документно-ориентирания модел. Конфигурациите притежават \textbf{йерархична структура} с вложени обекти, като например полета от типа \texttt{retryPolicy.maxRetries}, които представляват дълбоко вложени конфигурационни параметри. Те са \textbf{полуструктурирани}, което означава, че различни конфигурации могат да имат различни полета според специфичните си нужди. Изискват \textbf{атомарен достъп}, тъй като цялата конфигурация се чете и записва като единица. Всяка версия е \textbf{самостоятелен документ}, което позволява независимо управление и версиониране \cite{mongodb_indexing}.

\subsubsection{CAP теорема и MongoDB}

CAP теоремата дефинира трите основни свойства на разпределените системи: консистентност, наличност и толерантност към разделяния \cite{cap_theorem}. MongoDB позволява настройка на тези компромиси чрез параметрите write concern и read preference, давайки възможност за балансиране според конкретните изисквания.

\textbf{Консистентността} е конфигурируема, като write majority осигурява силна консистентност при запис, докато други настройки позволяват по-висока производителност. \textbf{Наличността} е висока чрез механизма replica sets, който осигурява автоматично възстановяване при отказ на сървър. \textbf{Толерантността към разделяния} се реализира чрез нативната поддръжка на sharding, която позволява разпределение на данните между множество сървъри.

За система за конфигурации, където четенията доминират над писанията, е оптимален балансът с eventual consistency за четения и strong consistency за критични операции по запис.

\subsection{Технологичен стек}

\subsubsection{Backend платформа}

\textbf{Node.js} е избран като сървърна платформа поради няколко ключови характеристики, които го правят идеален за интеграция с MongoDB \cite{nodejs_docs}. Платформата осигурява нативна JSON поддрържка, която съвпада идеално с BSON формата на MongoDB и позволява директна сериализация и десериализация на данни. Async I/O моделът е ефективен при операции, обвързани с вход и изход, като заявки към база данни, тъй като позволява на една нишка да обслужва множество едновременни връзки. Използването на единен език JavaScript в целия стек опростява разработката и поддръжката. Богатата екосистема от библиотеки включва Mongoose ODM, който осигурява високо ниво на абстракция при работа с MongoDB.

\textbf{Express.js} предоставя минималистичен, но мощен уеб фреймуърк \cite{express_docs}. Архитектурата му е базирана на middleware компоненти за cross-cutting concerns като логиране и аутентикация, които се изпълняват последователно при обработка на заявки. Router системата позволява структуриране на RESTful API с ясно разделение на маршрутите. Лесната интеграция с библиотеки за валидация и аутентикация опростява добавянето на допълнителна функционалност.

\subsubsection{Mongoose ODM}

Mongoose Object Document Mapper предоставя високо ниво на абстракция при работа с MongoDB \cite{mongoose_docs,banker2011mongodb}. Основните функционалности включват дефиниране на схема с валидация и автоматично преобразуване на типове, което осигурява цялостност на данните преди запис в базата. Middleware hooks позволяват изпълнение на код преди и след операции, като например изчисляване на контролна сума преди запис. Query building с метод chaining позволява конструиране на сложни заявки чрез свързване на методи. Population механизмът управлява релациите между документи и автоматично зареждане на свързани данни.

\subsubsection{Поддържащи технологии}

Освен основния технологичен стек, системата интегрира допълнителни технологии за специфични нужди, включително Docker за контейнеризация \cite{docker_docs}. Таблица 2.2 представя преглед на тези технологии.

\begin{table}[H]
\centering
\caption{Поддържащи технологии}
\begin{tabular}{|p{4cm}|p{9cm}|}
\hline
\textbf{Технология} & \textbf{Предназначение} \\
\hline
Docker & Контейнеризация за преносимо разгръщане \\
\hline
JWT & Stateless аутентикация между микросървиси \\
\hline
Helmet.js & Security headers за защита от чести уеб атаки \\
\hline
Joi & Схемна валидация за API входни данни \\
\hline
Winston & Структурирано логиране с множество транспорти \\
\hline
Rate-limiter-flexible & Защита срещу злоупотреба и DDoS атаки \\
\hline
\end{tabular}
\end{table}

\subsection{Сравнение с релационен подход}

\subsubsection{Проблеми при релационния модел}

Релационният подход за версиониране на конфигурации води до няколко значителни проблеми, които го правят неподходящ за този домейн.

\textbf{Entity-Attribute-Value анти-патернът} е често използван подход за съхраняване на полуструктурирани данни в релационни бази. Този модел обаче страда от сериозни недостатъци: загуба на type safety поради съхраняване на всички стойности като низове, необходимост от сложни self-join операции за реконструкция на целия обект, и лоша производителност при голяма дълбочина на вложеност.

\textbf{Версионирането} при релационни бази изисква компромиси между различни подходи, всеки със своите недостатъци. Копирането на цели редове при всяка нова версия води до излишно увеличаване на обема данни. Delta таблиците, съхраняващи само разликите, са сложни за заявки и изискват сложна логика за възстановяване на пълното състояние. JSON колоните предлагат частично решение, но с ограничена индексация и вторична поддръжка спрямо документно-ориентираните бази.

\textbf{Схемната ригидност} на релационния модел изисква миграции при промяна на структурата на конфигурацията, което е неприемливо за динамични системи с често променящи се изисквания. При документно-ориентираните бази промените в структурата се обработват автоматично без нужда от явни миграции.

\textbf{Join overhead} при възстановяване на пълна конфигурация с одитен запис изисква множество join операции между конфигурационни, версионни и одитни таблици, което деградира производителността при четене.

\subsubsection{Performance сравнение}

Таблица 2.3 представя сравнителен анализ на производителността между MongoDB и PostgreSQL с JSONB колони за конфигурационни данни.

\begin{table}[H]
\centering
\caption{Сравнение MongoDB vs PostgreSQL за конфигурационни данни}
\begin{tabular}{|p{4.5cm}|p{3.5cm}|p{4.5cm}|}
\hline
\textbf{Операция} & \textbf{MongoDB} & \textbf{PostgreSQL (JSONB)} \\
\hline
Read версия (indexed) & O(1) документ & O(1) ред + JSON parse \\
\hline
Write нова версия & O(1) insert & O(1) insert \\
\hline
Diff две версии & 2 документа & 2 реда + JSON parse \\
\hline
Query вложено поле & Native dot notation & JSONB operators \\
\hline
Horizontal scaling & Native sharding & Външни решения \\
\hline
\end{tabular}
\end{table}

Сравнението показва, че MongoDB предоставя по-добра производителност за специфичните операции, характерни за системи за версиониране на конфигурации, особено при мащабиране и работа с вложени структури.
